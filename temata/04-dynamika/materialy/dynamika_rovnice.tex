\documentclass[a4paper,11pt]{article}
\usepackage[utf8]{inputenc}
\usepackage[czech]{babel}
\usepackage{amsmath}
\usepackage{amssymb}
\usepackage{geometry}
\usepackage{xcolor}
\usepackage{tcolorbox}
\usepackage{fancyhdr}

\geometry{margin=2cm}

\tcbuselibrary{skins,breakable}

% Definice boxů
\newtcolorbox{definitionbox}{
    colback=green!5!white,
    colframe=green!50!black,
    fonttitle=\bfseries,
    breakable,
    enhanced
}

\newtcolorbox{equationbox}{
    colback=blue!5!white,
    colframe=blue!50!black,
    fonttitle=\bfseries,
    breakable,
    enhanced
}

\newtcolorbox{notebox}{
    colback=gray!5!white,
    colframe=gray!50!black,
    fonttitle=\bfseries,
    breakable,
    enhanced
}

\pagestyle{fancy}
\fancyhf{}
\rhead{Gymnázium Globe}
\lhead{Dynamika - Rovnice a definice}
\cfoot{\thepage}

\begin{document}

\begin{center}
    {\Huge \textbf{DYNAMIKA}}\\[0.5cm]
    {\Large Kompletní přehled rovnic a definic}\\[1cm]
\end{center}

\section{NEWTONOVY ZÁKONY POHYBU}

\subsection{1. Newtonův zákon (Zákon setrvačnosti)}

\begin{definitionbox}
Těleso setrvává v klidu nebo v rovnoměrném přímočarém pohybu, pokud není nuceno vnějšími silami tento stav změnit.
\end{definitionbox}

\subsection{2. Newtonův zákon (Zákon síly)}

\begin{definitionbox}
Zrychlení tělesa je přímo úměrné výsledné síle působící na těleso a nepřímo úměrné jeho hmotnosti.
\end{definitionbox}

\begin{equationbox}
\begin{equation*}
    F = m \cdot a
\end{equation*}
\end{equationbox}

\begin{notebox}
kde: $F$ [N] - síla, $m$ [kg] - hmotnost, $a$ [m$\cdot$s$^{-2}$] - zrychlení
\end{notebox}

\subsection{3. Newtonův zákon (Zákon akce a reakce)}

\begin{definitionbox}
Každé akci odpovídá stejně velká, ale opačně orientovaná reakce. Síly akce a reakce působí na různá tělesa.
\end{definitionbox}

\begin{equationbox}
\begin{equation*}
    F_{12} = -F_{21}
\end{equation*}
\end{equationbox}

\section{ZÁKLADNÍ SÍLY V DYNAMICE}

\subsection{Tíhová síla (Gravitační síla)}

\begin{equationbox}
\begin{equation*}
    F_g = m \cdot g
\end{equation*}
\end{equationbox}

\begin{notebox}
kde: $g = 9{,}81$ m$\cdot$s$^{-2}$ - tíhové zrychlení na Zemi
\end{notebox}

\subsection{Normálová síla}

\begin{definitionbox}
Kolmá síla od podložky, která brání průniku tělesa do podložky. Na vodorovné rovině: $F_n = F_g$
\end{definitionbox}

\subsection{Třecí síla}

\begin{equationbox}
\begin{equation*}
    F_t = f \cdot F_n
\end{equation*}
\end{equationbox}

\begin{notebox}
kde: $f$ - součinitel tření (bez jednotky), $F_n$ [N] - normálová síla
\end{notebox}

\begin{itemize}
    \item \textbf{Statické tření} - brání rozjetí tělesa
    \item \textbf{Kinetické tření} - působí při pohybu tělesa
\end{itemize}

\newpage

\section{NAKLONĚNÁ ROVINA}

\begin{definitionbox}
Na nakloněné rovině rozkládáme tíhovou sílu na dvě složky:
\end{definitionbox}

\subsection{Kolmá složka (na rovinu)}

\begin{equationbox}
\begin{equation*}
    F_{g\perp} = F_g \cdot \cos(\alpha) = m \cdot g \cdot \cos(\alpha)
\end{equation*}
\end{equationbox}

\subsection{Rovnoběžná složka (podél roviny)}

\begin{equationbox}
\begin{equation*}
    F_{g\parallel} = F_g \cdot \sin(\alpha) = m \cdot g \cdot \sin(\alpha)
\end{equation*}
\end{equationbox}

\begin{notebox}
kde: $\alpha$ [°] nebo [rad] - úhel sklonu roviny
\end{notebox}

\section{HYBNOST A IMPULZ SÍLY}

\subsection{Hybnost}

\begin{definitionbox}
Hybnost je vektorová veličina udávající míru pohybu tělesa.
\end{definitionbox}

\begin{equationbox}
\begin{equation*}
    p = m \cdot v
\end{equation*}
\end{equationbox}

\begin{notebox}
kde: $p$ [kg$\cdot$m$\cdot$s$^{-1}$] - hybnost, $m$ [kg] - hmotnost, $v$ [m$\cdot$s$^{-1}$] - rychlost
\end{notebox}

\subsection{Zákon zachování hybnosti}

\begin{definitionbox}
V izolované soustavě (bez vnějších sil) se celková hybnost zachovává.
\end{definitionbox}

\begin{equationbox}
\begin{equation*}
    p_1 + p_2 = p_1' + p_2'
\end{equation*}
\end{equationbox}

\begin{equationbox}
\begin{equation*}
    m_1 \cdot v_1 + m_2 \cdot v_2 = m_1 \cdot v_1' + m_2 \cdot v_2'
\end{equation*}
\end{equationbox}

\subsection{Impulz síly}

\begin{definitionbox}
Impulz síly je změna hybnosti tělesa.
\end{definitionbox}

\begin{equationbox}
\begin{equation*}
    I = F \cdot \Delta t = \Delta p = m \cdot \Delta v
\end{equation*}
\end{equationbox}

\begin{notebox}
kde: $I$ [N$\cdot$s] - impulz síly, $\Delta t$ [s] - doba působení síly
\end{notebox}

\newpage

\section{SRÁŽKY TĚLES}

\subsection{Nepružná srážka}

\begin{definitionbox}
Tělesa se po srážce spojí a pohybují se společnou rychlostí.
\end{definitionbox}

\begin{equationbox}
\begin{equation*}
    m_1 \cdot v_1 + m_2 \cdot v_2 = (m_1 + m_2) \cdot v'
\end{equation*}
\end{equationbox}

\subsection{Pružná srážka}

\begin{definitionbox}
Zachovává se hybnost i kinetická energie. Tělesa se po srážce odrazí.
\end{definitionbox}

\begin{equationbox}
\begin{equation*}
    m_1 \cdot v_1 + m_2 \cdot v_2 = m_1 \cdot v_1' + m_2 \cdot v_2'
\end{equation*}
\end{equationbox}

\begin{equationbox}
\begin{equation*}
    \frac{1}{2}m_1 \cdot v_1^2 + \frac{1}{2}m_2 \cdot v_2^2 = \frac{1}{2}m_1 \cdot v_1'^2 + \frac{1}{2}m_2 \cdot v_2'^2
\end{equation*}
\end{equationbox}

\section{POHYB PO KRUŽNICI}

\subsection{Dostředivá síla}

\begin{definitionbox}
Síla směřující ke středu kružnice, která udržuje těleso na kruhové dráze.
\end{definitionbox}

\begin{equationbox}
\begin{equation*}
    F_d = m \cdot a_d = m \cdot \frac{v^2}{r} = m \cdot \omega^2 \cdot r
\end{equation*}
\end{equationbox}

\begin{notebox}
kde: $r$ [m] - poloměr kružnice, $v$ [m$\cdot$s$^{-1}$] - rychlost, $\omega$ [rad$\cdot$s$^{-1}$] - úhlová rychlost
\end{notebox}

\subsection{Úhlová rychlost}

\begin{equationbox}
\begin{equation*}
    \omega = \frac{2\pi}{T} = 2\pi f
\end{equation*}
\end{equationbox}

\begin{notebox}
kde: $T$ [s] - perioda, $f$ [Hz] - frekvence
\end{notebox}

\section{UŽITEČNÉ VZTAHY}

\subsection{Převody jednotek}

\begin{notebox}
\begin{itemize}
    \item 1 N = 1 kg$\cdot$m$\cdot$s$^{-2}$
    \item 1 km$\cdot$h$^{-1}$ = 1/3,6 m$\cdot$s$^{-1}$
    \item 1 m$\cdot$s$^{-1}$ = 3,6 km$\cdot$h$^{-1}$
\end{itemize}
\end{notebox}

\subsection{Gravitační konstanta}

\begin{notebox}
$g = 9{,}81$ m$\cdot$s$^{-2}$ $\approx$ 10 m$\cdot$s$^{-2}$ (zaokrouhleno)
\end{notebox}

\vfill

\begin{center}
\rule{0.8\textwidth}{0.5pt}\\[0.3cm]
{\small Gymnázium Globe | Praktické aplikace fyziky a chemie}
\end{center}

\end{document}
