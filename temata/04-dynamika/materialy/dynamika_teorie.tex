\documentclass[11pt,a4paper]{article}
\usepackage[margin=2.5cm]{geometry}
\usepackage[T1]{fontenc}
\usepackage[utf8]{inputenc}
\usepackage[czech]{babel}
\usepackage{amsmath}
\usepackage{booktabs}
\usepackage{longtable}

\title{Dynamika - Teoretický přehled}
\author{Fyzika - opakování a prohloubení}
\date{}

\begin{document}
\maketitle

\section{Newtonovy zákony pohybu}

\subsection{1. Newtonův zákon (Zákon setrvačnosti)}

\textbf{Formulace:}
\begin{quote}
Těleso setrvává v klidu nebo v rovnoměrném přímočarém pohybu, pokud není nuceno vnějšími silami tento stav změnit.
\end{quote}

\textbf{Fyzikální význam:}
\begin{itemize}
\item Každé těleso má vlastnost setrvačnosti
\item Setrvačnost závisí na hmotnosti
\item Platí v inerciálních vztažných soustavách
\end{itemize}

\subsection{2. Newtonův zákon (Zákon síly)}

\textbf{Rovnice:}
\[\vec{F} = m \cdot \vec{a}\]

nebo

\[\vec{a} = \frac{\vec{F}}{m}\]

\textbf{Popis veličin:}

\begin{longtable}{lll}
\toprule
Veličina & Popis & Jednotka \\
\midrule
$\vec{F}$ & Síla (vektor) & N (Newton) \\
$m$ & Hmotnost & kg \\
$\vec{a}$ & Zrychlení (vektor) & m/s² \\
\bottomrule
\end{longtable}

\textbf{Fyzikální význam:}
\begin{itemize}
\item Síla způsobuje zrychlení
\item Zrychlení je přímo úměrné síle a nepřímo úměrné hmotnosti
\item Směr zrychlení je stejný jako směr výsledné síly
\end{itemize}

\subsection{3. Newtonův zákon (Zákon akce a reakce)}

\textbf{Formulace:}
\begin{quote}
Každé působení (akce) vyvolává stejně velké opačné působení (reakci).
\end{quote}

\textbf{Rovnice:}
\[\vec{F}_{12} = -\vec{F}_{21}\]

\textbf{Fyzikální význam:}
\begin{itemize}
\item Síly vznikají vždy ve dvojicích
\item Akce a reakce působí na různá tělesa
\item Akce a reakce mají stejnou velikost a opačný směr
\end{itemize}

\clearpage

\section{Druhy sil}

\subsection{Tíhová síla}

\textbf{Rovnice:}
\[F_g = m \cdot g\]

\textbf{Popis veličin:}

\begin{longtable}{lll}
\toprule
Veličina & Popis & Jednotka \\
\midrule
$F_g$ & Tíhová síla & N \\
$m$ & Hmotnost & kg \\
$g$ & Tíhové zrychlení & m/s² \\
\bottomrule
\end{longtable}

\textbf{Fyzikální význam:}
\begin{itemize}
\item Gravitační síla působící na těleso v blízkosti Země
\item Na Zemi: $g \approx 9{,}81$ m/s²
\item Směřuje svisle dolů k středu Země
\end{itemize}

\subsection{Třecí síla}

\textbf{Rovnice:}
\[F_t = f \cdot F_n\]

\textbf{Popis veličin:}

\begin{longtable}{lll}
\toprule
Veličina & Popis & Jednotka \\
\midrule
$F_t$ & Třecí síla & N \\
$f$ & Součinitel tření & - \\
$F_n$ & Normálová síla & N \\
\bottomrule
\end{longtable}

\textbf{Druhy tření:}
\begin{itemize}
\item \textbf{Statické tření} ($f_s$) - těleso v klidu
\item \textbf{Kinetické tření} ($f_k$) - těleso v pohybu
\item Platí: $f_s > f_k$
\end{itemize}

\subsection{Dostředivá síla}

\textbf{Rovnice:}
\[F_d = m \cdot \frac{v^2}{r} = m \cdot \omega^2 \cdot r\]

\textbf{Popis veličin:}

\begin{longtable}{lll}
\toprule
Veličina & Popis & Jednotka \\
\midrule
$F_d$ & Dostředivá síla & N \\
$m$ & Hmotnost & kg \\
$v$ & Obvodová rychlost & m/s \\
$r$ & Poloměr & m \\
$\omega$ & Úhlová rychlost & rad/s \\
\bottomrule
\end{longtable}

\textbf{Fyzikální význam:}
\begin{itemize}
\item Výslednice sil při pohybu po kružnici
\item Směřuje k středu kružnice
\item Nutná pro udržení kruhové trajektorie
\end{itemize}

\clearpage

\section{Síla na nakloněné rovině}

\subsection{Rozklad tíhové síly}

\textbf{Rovnoběžná složka (po nakloněné rovině):}
\[F_{||} = m \cdot g \cdot \sin \alpha\]

\textbf{Kolmá složka (na nakloněnou rovinu):}
\[F_{\perp} = m \cdot g \cdot \cos \alpha\]

\textbf{Popis veličin:}

\begin{longtable}{lll}
\toprule
Veličina & Popis & Jednotka \\
\midrule
$F_{||}$ & Síla rovnoběžná s rovinou & N \\
$F_{\perp}$ & Síla kolmá k rovině & N \\
$\alpha$ & Úhel sklonu roviny & ° nebo rad \\
\bottomrule
\end{longtable}

\textbf{Normálová síla:}
\[F_n = F_{\perp} = m \cdot g \cdot \cos \alpha\]

\textbf{Podmínka rovnováhy (bez tření):}
\[F_{tah} = F_{||} = m \cdot g \cdot \sin \alpha\]

\textbf{S třením:}
\[F_{tah} = m \cdot g \cdot \sin \alpha + f \cdot m \cdot g \cdot \cos \alpha\]

\clearpage

\section{Hybnost a zákony zachování}

\subsection{Hybnost}

\textbf{Definice:}
\begin{quote}
Hybnost je vektorová veličina rovná součinu hmotnosti a rychlosti.
\end{quote}

\textbf{Rovnice:}
\[\vec{p} = m \cdot \vec{v}\]

\textbf{Popis veličin:}

\begin{longtable}{lll}
\toprule
Veličina & Popis & Jednotka \\
\midrule
$\vec{p}$ & Hybnost & kg·m/s \\
$m$ & Hmotnost & kg \\
$\vec{v}$ & Rychlost & m/s \\
\bottomrule
\end{longtable}

\subsection{Zákon zachování hybnosti}

\textbf{Formulace:}
\begin{quote}
V izolované soustavě je celková hybnost konstantní.
\end{quote}

\textbf{Rovnice:}
\[\sum \vec{p}_{\text{před}} = \sum \vec{p}_{\text{po}}\]

Pro dvě tělesa:
\[m_1 \vec{v}_1 + m_2 \vec{v}_2 = m_1 \vec{v}'_1 + m_2 \vec{v}'_2\]

\textbf{Aplikace:}
\begin{itemize}
\item Srážky těles (pružné i nepružné)
\item Výstřel z děla (zpětný ráz)
\item Raketový pohon
\end{itemize}

\subsection{Impuls síly}

\textbf{Definice:}
\begin{quote}
Impuls síly je změna hybnosti.
\end{quote}

\textbf{Rovnice:}
\[\Delta \vec{p} = \vec{F} \cdot \Delta t\]

nebo
\[\vec{F} = \frac{\Delta \vec{p}}{\Delta t}\]

\textbf{Fyzikální význam:}
\begin{itemize}
\item Stejná změna hybnosti může být dosažena malou silou po dlouhý čas nebo velkou silou po krátký čas
\item Airbag v autě prodlužuje dobu nárazu → menší síla
\end{itemize}

\clearpage

\section{Fyzikální konstanty}

\begin{longtable}{llll}
\toprule
Konstanta & Symbol & Hodnota & Jednotka \\
\midrule
Tíhové zrychlení (Země) & $g$ & 9,81 & m/s² \\
Tíhové zrychlení (Měsíc) & $g_M$ & 1,62 & m/s² \\
\bottomrule
\end{longtable}

\subsection{Typické hodnoty součinitelů tření}

\begin{longtable}{lll}
\toprule
Materiály & Statické $f_s$ & Kinetické $f_k$ \\
\midrule
Ocel - ocel & 0,74 & 0,57 \\
Dřevo - dřevo & 0,25-0,50 & 0,20 \\
Led - led & 0,10 & 0,03 \\
Guma - suchý asfalt & 1,0 & 0,8 \\
Guma - mokrý asfalt & 0,7 & 0,5 \\
\bottomrule
\end{longtable}

\clearpage

\section{Souhrn jednotek v SI}

\begin{longtable}{llll}
\toprule
Veličina & Jednotka SI & Odvození & Poznámka \\
\midrule
Síla & N (Newton) & kg·m/s² & 1 N = síla potřebná k udělení zrychlení 1 m/s² tělesu o hmotnosti 1 kg \\
Hmotnost & kg (kilogram) & - & Základní jednotka \\
Zrychlení & m/s² & - & - \\
Hybnost & kg·m/s & - & Alternativně N·s \\
Impuls síly & N·s & - & Alternativně kg·m/s \\
\bottomrule
\end{longtable}

\clearpage

\section*{Poznámky}

\begin{itemize}
\item \textbf{Inerciální soustava:} Vztažná soustava, ve které platí Newtonovy zákony (soustava pohybující se rovnoměrně přímočaře nebo v klidu)
\item \textbf{Setrvačná hmotnost vs. tíhová hmotnost:} V klasické mechanice jsou totožné, obecně mohou být různé
\item \textbf{Výsledná síla:} Vektorový součet všech sil působících na těleso
\item \textbf{Rovnováha sil:} $\sum \vec{F} = 0$ → těleso je v klidu nebo se pohybuje rovnoměrně přímočaře
\item \textbf{Třecí síla:} Vždy působí proti směru pohybu (nebo možného pohybu)
\item \textbf{Izolovaná soustava:} Soustava, na kterou nepůsobí vnější síly (uzavřená soustava)
\item \textbf{Pružná srážka:} Zachovává se hybnost i kinetická energie
\item \textbf{Nepružná srážka:} Zachovává se pouze hybnost, ne kinetická energie
\end{itemize}

\end{document}