\documentclass[11pt,a4paper]{article}
\usepackage[margin=2.5cm]{geometry}
\usepackage[T1]{fontenc}
\usepackage[utf8]{inputenc}
\usepackage[czech]{babel}
\usepackage{amsmath}
\usepackage{booktabs}
\usepackage{longtable}

\title{Úvod a bilance - Teoretický přehled}
\author{Fyzika - opakování a prohloubení}
\date{}

\begin{document}
\maketitle

\section{Základní pojmy}

\subsection{Teplota (Temperature)}

\textbf{Definice:}
\begin{quote}
Teplota je stavová veličina vyjadřující průměrnou kinetickou energii molekul látky. Je nezávislá na množství látky.
\end{quote}

\textbf{Jednotky:}
\begin{itemize}
\item \textbf{°C} (stupeň Celsia) - nejběžnější v Evropě
\item \textbf{K} (Kelvin) - základní jednotka SI, absolutní teplota
\item \textbf{°F} (stupeň Fahrenheita) - používá se v USA
\end{itemize}

\textbf{Převody:}
\[T[K] = T[°C] + 273{,}15\]
\[T[°F] = T[°C] \times \frac{9}{5} + 32\]

\subsection{Teplo (Heat)}

\textbf{Definice:}
\begin{quote}
Teplo je dějová veličina - forma energie, která se přenáší mezi tělesy s různou teplotou. Závisí na množství látky, teplotním rozdílu a druhu látky.
\end{quote}

\textbf{Jednotka:} J (Joule)

\subsection{Tepelný tok (Heat Flow)}

\textbf{Definice:}
\begin{quote}
Tepelný tok udává množství tepla přenášeného za jednotku času.
\end{quote}

\textbf{Rovnice:}
\[\dot{Q} = \frac{Q}{t}\]

\textbf{Jednotka:} W (Watt) = J/s

\subsection{Entalpie (Enthalpy)}

\textbf{Definice:}
\begin{quote}
Entalpie je fyzikální veličina rozměru energie, která vyjadřuje celkovou energii systému včetně vnitřní energie a energie spojené s tlakem a objemem.
\end{quote}

\textbf{Symbol:} H \\
\textbf{Jednotka:} J (Joule)

\clearpage

\section{Základní rovnice}

\subsection{Diskontinuální provoz (bez průtoku v čase)}

\textbf{Ohřev nebo ochlazení látky:}
\[Q = m \cdot c_p \cdot (T_2 - T_1)\]

\textbf{Popis veličin:}

\begin{longtable}{lll}
\toprule
Veličina & Popis & Jednotka \\
\midrule
$Q$ & Teplo & J (Joule) \\
$m$ & Hmotnost & kg \\
$c_p$ & Měrná tepelná kapacita & J/(kg·K) \\
$T_1$ & Počáteční teplota & °C nebo K \\
$T_2$ & Konečná teplota & °C nebo K \\
\bottomrule
\end{longtable}

\subsection{Kontinuální provoz (s průtokem v čase)}

\textbf{Ohřev nebo ochlazení tekutiny:}
\[\dot{Q} = \dot{m} \cdot c_p \cdot (T_2 - T_1)\]

\textbf{Popis veličin:}

\begin{longtable}{lll}
\toprule
Veličina & Popis & Jednotka \\
\midrule
$\dot{Q}$ & Tepelný tok & W (Watt) \\
$\dot{m}$ & Hmotnostní tok & kg/s \\
$c_p$ & Měrná tepelná kapacita & J/(kg·K) \\
$T_1$ & Teplota na vstupu & °C nebo K \\
$T_2$ & Teplota na výstupu & °C nebo K \\
\bottomrule
\end{longtable}

\subsection{S fázovou změnou (var/kondenzace)}

\textbf{Kompletní rovnice s odpařováním:}
\[\dot{Q} = \dot{m} \cdot c_{p,kapaliny} \cdot (T_{var} - T_1) + \dot{m} \cdot l_v + \dot{m} \cdot c_{p,páry} \cdot (T_2 - T_{var})\]

\textbf{Popis veličin:}

\begin{longtable}{lll}
\toprule
Veličina & Popis & Jednotka \\
\midrule
$c_{p,kapaliny}$ & Měrná tepelná kapacita kapaliny & J/(kg·K) \\
$c_{p,páry}$ & Měrná tepelná kapacita páry & J/(kg·K) \\
$l_v$ & Měrné skupenské teplo varu & J/kg \\
$T_{var}$ & Teplota varu & °C nebo K \\
\bottomrule
\end{longtable}

\textbf{Fyzikální význam:}
\begin{itemize}
\item \textbf{První člen:} Ohřev kapaliny na teplotu varu
\item \textbf{Druhý člen:} Energie potřebná ke změně skupenství (odpařování)
\item \textbf{Třetí člen:} Ohřev páry nad teplotu varu
\end{itemize}

\clearpage

\section{Zákony zachování}

\subsection{Zákon zachování hmoty}

\textbf{Obecná bilance:}
\[\text{Vstup} = \text{Výstup} + \text{Akumulace}\]

\textbf{Pro ustálený stav (akumulace = 0):}
\[\dot{m}_{vstup} = \dot{m}_{výstup}\]

\textbf{Fyzikální význam:}
\begin{itemize}
\item Hmota nemůže vzniknout ani zaniknout
\item V uzavřeném systému je celková hmotnost konstantní
\item Hmotnostní toky do systému se rovnají hmotnostním tokům ze systému
\end{itemize}

\subsection{Zákon zachování energie}

\textbf{První termodynamický zákon:}
\[\text{Energie vstupující} - \text{Energie vystupující} = \text{Akumulace energie}\]

\textbf{Pro ustálený stav:}
\[\dot{Q}_{vstup} + \dot{W}_{vstup} = \dot{Q}_{výstup} + \dot{W}_{výstup}\]

\textbf{Fyzikální význam:}
\begin{itemize}
\item Energie nemůže vzniknout ani zaniknout, pouze se přeměňuje
\item Celková energie izolovaného systému je konstantní
\item Energetická bilance zahrnuje teplo i práci
\end{itemize}

\subsection{Obecná bilance}

\textbf{Univerzální rovnice bilance:}
\[\text{Vstup} + \text{Výroba} = \text{Výstup} + \text{Spotřeba} + \text{Akumulace}\]

\textbf{Aplikace:}
\begin{itemize}
\item \textbf{Hmotnostní bilance:} Výroba = 0, Spotřeba = 0
\item \textbf{Energetická bilance:} Energie se nemění na hmotu
\item \textbf{Látkové bilance:} Pro jednotlivé složky směsi
\end{itemize}

\clearpage

\section{Měrné tepelné kapacity běžných látek}

\begin{longtable}{lll}
\toprule
Látka & $c_p$ [J/(kg·K)] & Poznámka \\
\midrule
Voda (kapalná) & 4\,186 & Nejvyšší ze společných látek \\
Vzduch & 1\,005 & Při konstantním tlaku \\
Led & 2\,050 & Při 0°C \\
Vodní pára & 2\,010 & Při 100°C \\
Hliník & 897 & Kovy obecně nízká hodnota \\
Měď & 385 & \\
Železo & 449 & \\
Olovo & 129 & \\
Ethanol & 2\,440 & \\
Rtuť & 140 & \\
\bottomrule
\end{longtable}

\textbf{Důležité poznatky:}
\begin{itemize}
\item Voda má jednu z nejvyšších měrných tepelných kapacit
\item Kovy mají obecně nízké hodnoty $c_p$
\item Vyšší $c_p$ znamená větší schopnost akumulovat teplo
\end{itemize}

\clearpage

\section{Měrná skupenská tepla}

\subsection{Voda}

\begin{longtable}{llll}
\toprule
Děj & Symbol & Hodnota & Jednotka \\
\midrule
Tání ledu & $l_t$ & 334\,000 & J/kg \\
Tuhnutí vody & $l_t$ & 334\,000 & J/kg \\
Var vody & $l_v$ & 2\,260\,000 & J/kg \\
Kondenzace páry & $l_v$ & 2\,260\,000 & J/kg \\
\bottomrule
\end{longtable}

\textbf{Fyzikální význam:}
\begin{itemize}
\item Skupenské teplo je energie potřebná ke změně skupenství bez změny teploty
\item Při varu a kondenzaci se přenáší velké množství energie
\item Proto je vodní pára efektivní pro přenos tepla (parní topení)
\end{itemize}

\subsection{Další látky}

\begin{longtable}{lll}
\toprule
Látka & Tání $l_t$ [J/kg] & Var $l_v$ [J/kg] \\
\midrule
Ethanol & 108\,000 & 838\,000 \\
Hliník & 397\,000 & 10\,900\,000 \\
Olovo & 23\,000 & 858\,000 \\
Rtuť & 11\,800 & 272\,000 \\
\bottomrule
\end{longtable}

\clearpage

\section{Praktické aplikace}

\subsection{Výpočet tepla pro ohřev vody}

\textbf{Bez fázové změny:}
\[Q = m \cdot c_p \cdot \Delta T\]

\textbf{Příklad:} Ohřát 2 kg vody z 20°C na 80°C
\[Q = 2 \cdot 4186 \cdot (80-20) = 502{,}320 \text{ J} = 502{,}3 \text{ kJ}\]

\subsection{Výpočet výkonu pro var vody}

\textbf{S fázovou změnou:}
\[\dot{Q} = \dot{m} \cdot (c_p \cdot \Delta T + l_v)\]

\textbf{Příklad:} Odpařit 0,1 kg/s vody při 100°C (již horká voda)
\[\dot{Q} = 0{,}1 \cdot 2{,}26 \times 10^6 = 226{,}000 \text{ W} = 226 \text{ kW}\]

\subsection{Směšování látek}

\textbf{Zákon zachování energie při směšování:}
\[m_1 \cdot c_{p1} \cdot (T_1 - T_{výsl}) = m_2 \cdot c_{p2} \cdot (T_{výsl} - T_2)\]

Kde:
\begin{itemize}
\item $T_1$ > $T_{výsl}$ > $T_2$
\item Teplejší látka odevzdává teplo, chladnější přijímá
\item Výsledná teplota $T_{výsl}$ leží mezi $T_1$ a $T_2$
\end{itemize}

\clearpage

\section{Souhrn jednotek v SI}

\begin{longtable}{llll}
\toprule
Veličina & Jednotka SI & Další jednotky & Převody \\
\midrule
Teplota & K (Kelvin) & °C, °F & K = °C + 273,15 \\
Teplo & J (Joule) & kJ, MJ, cal & 1 cal = 4,186 J \\
Tepelný tok & W (Watt) & kW, MW & 1 W = 1 J/s \\
Entalpie & J (Joule) & kJ, MJ & - \\
Hmotnost & kg (kilogram) & g, t & 1 t = 1000 kg \\
Hmotnostní tok & kg/s & kg/h, t/h & 1 kg/h = 0,000278 kg/s \\
Měrná tepelná kapacita & J/(kg·K) & kJ/(kg·K) & - \\
Měrné skupenské teplo & J/kg & kJ/kg, MJ/kg & - \\
Čas & s (sekunda) & min, h & 1 h = 3600 s \\
\bottomrule
\end{longtable}

\clearpage

\section*{Poznámky}

\begin{itemize}
\item \textbf{Teplota vs. Teplo:} Teplota je stavová veličina (nezávisí na množství), teplo je dějová (závisí na množství)
\item \textbf{Uzavřený systém:} Systém, který nevyměňuje hmotu s okolím (ale může vyměňovat energii)
\item \textbf{Izolovaný systém:} Systém, který nevyměňuje ani hmotu, ani energii s okolím
\item \textbf{Ustálený stav:} Stav, kdy se veličiny v systému nemění v čase (akumulace = 0)
\item \textbf{Fázová změna:} Změna skupenství probíhá při konstantní teplotě (bod varu, bod tání)
\item \textbf{Měrná vs. celková veličina:} Měrná veličina je vztažena na jednotku hmotnosti (J/kg), celková na celkové množství (J)
\end{itemize}

\end{document}