\documentclass[11pt,a4paper]{article}
\usepackage[margin=2.5cm]{geometry}
\usepackage[T1]{fontenc}
\usepackage[utf8]{inputenc}
\usepackage[czech]{babel}
\usepackage{amsmath}
\usepackage{booktabs}
\usepackage{longtable}

\title{Gravitační pole - Teoretický přehled}
\author{Fyzika - opakování a prohloubení}
\date{}

\begin{document}
\maketitle

\section{Newtonův gravitační zákon}

\subsection{Zákon univerzální gravitace}

\textbf{Rovnice:}
\[F_g = \kappa \frac{m_1 m_2}{r^2}\]

\textbf{Popis veličin:}

\begin{longtable}{lll}
\toprule
Veličina & Popis & Jednotka \\
\midrule
$F_g$ & Gravitační síla & N (newton) \\
$\kappa$ & Gravitační konstanta & N·m²·kg⁻² \\
$m_1, m_2$ & Hmotnosti těles & kg (kilogram) \\
$r$ & Vzdálenost středů těles & m (metr) \\
\bottomrule
\end{longtable}

\textbf{Hodnota gravitační konstanty:}
\[\kappa = 6{,}674 \times 10^{-11} \text{ N·m}^2\text{·kg}^{-2}\]

\textbf{Důležité vlastnosti:}

\begin{itemize}
\item Gravitační síla je vždy \textbf{přitažlivá}
\item Síla klesá s \textbf{druhou mocninou} vzdálenosti
\item Síla působí podél spojnice středů těles
\item Gravitace je \textbf{univerzální} - působí mezi všemi tělesy s hmotností
\end{itemize}

\clearpage

\section{Gravitační pole}

\subsection{Intenzita gravitačního pole}

\textbf{Rovnice:}
\[K = \frac{F_g}{m} = \kappa \frac{M}{r^2}\]

\textbf{Popis veličin:}

\begin{longtable}{lll}
\toprule
Veličina & Popis & Jednotka \\
\midrule
$K$ & Intenzita gravitačního pole & N/kg nebo m/s² \\
$F_g$ & Gravitační síla & N \\
$m$ & Hmotnost zkušebního tělesa & kg \\
$M$ & Hmotnost tělesa vytvářejícího pole & kg \\
$r$ & Vzdálenost od středu pole & m \\
\bottomrule
\end{longtable}

\textbf{Fyzikální význam:}

\begin{itemize}
\item Intenzita pole udává \textbf{sílu na jednotku hmotnosti}
\item Číselně rovna \textbf{gravitačnímu zrychlení} v daném místě
\item Nezávisí na hmotnosti zkušebního tělesa
\end{itemize}

\subsection{Gravitační zrychlení}

\textbf{Rovnice:}
\[a_g = K = \kappa \frac{M}{r^2}\]

\textbf{Na povrchu Země:}
\[g = \kappa \frac{M_Z}{R_Z^2} \approx 9{,}81 \text{ m/s}^2\]

\textbf{Popis veličin:}

\begin{longtable}{lll}
\toprule
Veličina & Popis & Jednotka \\
\midrule
$a_g$ & Gravitační zrychlení & m/s² \\
$g$ & Tíhové zrychlení na Zemi & m/s² \\
$M_Z$ & Hmotnost Země & kg \\
$R_Z$ & Poloměr Země & m \\
\bottomrule
\end{longtable}

\clearpage

\section{Gravitační potenciál}

\textbf{Rovnice:}
\[\varphi = -\kappa \frac{M}{r}\]

\textbf{Popis veličin:}

\begin{longtable}{lll}
\toprule
Veličina & Popis & Jednotka \\
\midrule
$\varphi$ & Gravitační potenciál & J/kg \\
$M$ & Hmotnost tělesa & kg \\
$r$ & Vzdálenost od středu & m \\
\bottomrule
\end{longtable}

\textbf{Fyzikální význam:}

\begin{itemize}
\item Potenciál je \textbf{záporný} (gravitace je přitažlivá síla)
\item Udává \textbf{práci na jednotku hmotnosti} potřebnou k přenesení tělesa z daného místa do nekonečna
\item Souvisí s potenciální energií: $E_p = m\varphi$
\end{itemize}

\clearpage

\section{Kosmické rychlosti}

\subsection{Kruhová rychlost (1. kosmická rychlost)}

\textbf{Rovnice:}
\[v_k = \sqrt{\kappa \frac{M}{r}}\]

\textbf{Pro nízké oběžné dráhy kolem Země:}
\[v_k \approx 7{,}9 \text{ km/s}\]

\textbf{Popis veličin:}

\begin{longtable}{lll}
\toprule
Veličina & Popis & Jednotka \\
\midrule
$v_k$ & Kruhová rychlost & m/s \\
$M$ & Hmotnost centrálního tělesa & kg \\
$r$ & Poloměr oběžné dráhy & m \\
\bottomrule
\end{longtable}

\textbf{Fyzikální význam:}

\begin{itemize}
\item Rychlost potřebná pro \textbf{kruhovou oběžnou dráhu}
\item Dostředivá síla se rovná gravitační síle
\item Pro nízké dráhy kolem Země: $v_k \approx 7{,}9$ km/s
\end{itemize}

\subsection{Úniková rychlost (2. kosmická rychlost)}

\textbf{Rovnice:}
\[v_u = \sqrt{2\kappa \frac{M}{r}} = v_k\sqrt{2}\]

\textbf{Z povrchu Země:}
\[v_u \approx 11{,}2 \text{ km/s}\]

\textbf{Popis veličin:}

\begin{longtable}{lll}
\toprule
Veličina & Popis & Jednotka \\
\midrule
$v_u$ & Úniková rychlost & m/s \\
$M$ & Hmotnost tělesa & kg \\
$r$ & Vzdálenost od středu tělesa & m \\
\bottomrule
\end{longtable}

\textbf{Fyzikální význam:}

\begin{itemize}
\item Minimální rychlost k \textbf{opuštění gravitačního pole}
\item Kinetická energie se rovná potenciální energii
\item Úniková rychlost je $\sqrt{2}$-krát větší než kruhová rychlost
\end{itemize}

\subsection{Další kosmické rychlosti}

\textbf{3. kosmická rychlost} (únik ze Sluneční soustavy):
\[v_3 \approx 16{,}7 \text{ km/s}\]

\textbf{4. kosmická rychlost} (únik z Galaxie):
\[v_4 \approx 525 \text{ km/s}\]

\clearpage

\section{Keplerovy zákony}

\subsection{1. Keplerův zákon (zákon elips)}

\textbf{Formulace:}

\begin{quote}
Planety se pohybují po \textbf{elipsách}, v jejichž jednom \textbf{ohnisku} je Slunce.
\end{quote}

\textbf{Matematický popis:}

\begin{itemize}
\item Dráha planety je elipsa s hlavní poloosou $a$ a vedlejší poloosou $b$
\item Slunce je v jednom z ohnisek elipsy
\item Vzdálenost mezi ohnisky: $2c$, kde $c = \sqrt{a^2 - b^2}$
\end{itemize}

\subsection{2. Keplerův zákon (zákon ploch)}

\textbf{Formulace:}

\begin{quote}
Plocha opsaná \textbf{průvodičem} planety je za stejný čas \textbf{stejná}.
\end{quote}

\textbf{Fyzikální důsledek:}

\begin{itemize}
\item Planeta se pohybuje \textbf{rychleji}, když je blíže Slunci (v \textbf{perihelu})
\item Planeta se pohybuje \textbf{pomaleji}, když je dále od Slunce (v \textbf{afelu})
\item Zachování \textbf{momentu hybnosti}
\end{itemize}

\subsection{3. Keplerův zákon (harmonický zákon)}

\textbf{Rovnice:}
\[\frac{T_1^2}{T_2^2} = \frac{a_1^3}{a_2^3}\]

nebo pro jednu planetu:
\[T^2 = \frac{4\pi^2}{GM} a^3\]

\textbf{Popis veličin:}

\begin{longtable}{lll}
\toprule
Veličina & Popis & Jednotka \\
\midrule
$T$ & Oběžná doba planety & s (sekunda) \\
$a$ & Hlavní poloosa dráhy & m \\
$G$ & Gravitační konstanta ($\kappa$) & N·m²·kg⁻² \\
$M$ & Hmotnost centrálního tělesa & kg \\
\bottomrule
\end{longtable}

\textbf{Fyzikální význam:}

\begin{itemize}
\item Druhá mocnina oběžné doby je úměrná třetí mocnině hlavní poloosy
\item Platí pro všechna tělesa obíhající kolem stejného centrálního tělesa
\end{itemize}

\clearpage

\section{Fyzikální konstanty}

\begin{longtable}{llll}
\toprule
Konstanta & Symbol & Hodnota & Jednotka \\
\midrule
Gravitační konstanta & $\kappa$ nebo $G$ & $6{,}674 \times 10^{-11}$ & N·m²·kg⁻² \\
Hmotnost Země & $M_Z$ & $5{,}972 \times 10^{24}$ & kg \\
Poloměr Země & $R_Z$ & $6{,}371 \times 10^{6}$ & m \\
Hmotnost Slunce & $M_S$ & $1{,}989 \times 10^{30}$ & kg \\
Hmotnost Měsíce & $M_M$ & $7{,}342 \times 10^{22}$ & kg \\
Vzdálenost Země-Měsíc & $r_{ZM}$ & $3{,}844 \times 10^{8}$ & m \\
Tíhové zrychlení na Zemi & $g$ & $9{,}81$ & m/s² \\
\bottomrule
\end{longtable}

\clearpage

\section{Užitečné vztahy a vzorce}

\subsection{Energie v gravitačním poli}

\textbf{Potenciální energie:}
\[E_p = -\kappa \frac{Mm}{r}\]

\textbf{Kinetická energie kruhové oběžné dráhy:}
\[E_k = \frac{1}{2}mv_k^2 = \kappa \frac{Mm}{2r}\]

\textbf{Celková mechanická energie:}
\[E = E_k + E_p = -\kappa \frac{Mm}{2r}\]

\subsection{Vztah mezi zrychlením a výškou}

\textbf{Gravitační zrychlení ve výšce $h$ nad povrchem:}
\[g_h = g \frac{R^2}{(R+h)^2}\]

kde:
\begin{itemize}
\item $g$ je tíhové zrychlení na povrchu
\item $R$ je poloměr tělesa
\item $h$ je výška nad povrchem
\end{itemize}

\subsection{Podmínky pro oběžné dráhy}

\textbf{Kruhová dráha:}
\[F_g = F_{ds} \quad \Rightarrow \quad \kappa \frac{Mm}{r^2} = m\frac{v^2}{r}\]

\textbf{Eliptická dráha:}
\[v_k < v < v_u\]

\textbf{Únik z gravitačního pole:}
\[v \geq v_u\]

\clearpage

\section{Souhrn jednotek v SI}

\begin{longtable}{lll}
\toprule
Veličina & Jednotka SI & Další jednotky \\
\midrule
Síla & N (newton) & kg·m/s² \\
Hmotnost & kg (kilogram) & - \\
Vzdálenost & m (metr) & km = 10³ m \\
Rychlost & m/s & km/s, km/h \\
Zrychlení & m/s² & - \\
Energie & J (joule) & N·m, kg·m²/s² \\
Potenciál & J/kg & m²/s² \\
Intenzita pole & N/kg & m/s² \\
Čas & s (sekunda) & h (hodina), den, rok \\
\bottomrule
\end{longtable}

\clearpage

\section*{Poznámky}

\begin{itemize}
\item Všechny vzorce předpokládají \textbf{bodové hmotnosti} nebo \textbf{kulově symetrická tělesa}
\item Vzdálenost $r$ se měří od \textbf{středu těles}
\item V reálných situacích může být třeba zohlednit \textbf{atmosféru} a \textbf{odpor vzduchu}
\item Pro velmi přesné výpočty je nutné použít \textbf{relativistickou mechaniku}
\end{itemize}

\end{document}