\documentclass[a4paper,11pt]{article}
\usepackage[utf8]{inputenc}
\usepackage[czech]{babel}
\usepackage{amsmath}
\usepackage{amssymb}
\usepackage{geometry}
\usepackage{xcolor}
\usepackage{tcolorbox}
\usepackage{fancyhdr}

\geometry{margin=2cm}

\tcbuselibrary{skins,breakable}

% Definice boxů
\newtcolorbox{definitionbox}{
    colback=green!5!white,
    colframe=green!50!black,
    fonttitle=\bfseries,
    breakable,
    enhanced
}

\newtcolorbox{equationbox}{
    colback=blue!5!white,
    colframe=blue!50!black,
    fonttitle=\bfseries,
    breakable,
    enhanced
}

\newtcolorbox{notebox}{
    colback=gray!5!white,
    colframe=gray!50!black,
    fonttitle=\bfseries,
    breakable,
    enhanced
}

\pagestyle{fancy}
\fancyhf{}
\rhead{Gymnázium Globe}
\lhead{Mechanická práce a energie - Rovnice a definice}
\cfoot{\thepage}

\begin{document}

\begin{center}
    {\Huge \textbf{MECHANICKÁ PRÁCE A ENERGIE}}\\[0.5cm]
    {\Large Kompletní přehled rovnic a definic}\\[1cm]
\end{center}

\section{MECHANICKÁ PRÁCE}

\subsection{Definice mechanické práce}

\begin{definitionbox}
Mechanická práce je fyzikální veličina charakterizující děj, při kterém se přemísťují tělesa vlivem působení síly.
\end{definitionbox}

\begin{equationbox}
\begin{equation*}
    W = F \cdot s \cdot \cos\alpha
\end{equation*}
\end{equationbox}

\begin{notebox}
kde: $W$ [J] - práce (Joule), $F$ [N] - síla, $s$ [m] - dráha, $\alpha$ [°] - úhel mezi silou a směrem pohybu
\end{notebox}

\subsection{Speciální případy}

\begin{notebox}
\begin{itemize}
    \item $\alpha = 0°$: $\cos 0° = 1$ $\Rightarrow$ $W = F \cdot s$ (kladná práce)
    \item $\alpha = 90°$: $\cos 90° = 0$ $\Rightarrow$ $W = 0$ (žádná práce)
    \item $\alpha = 180°$: $\cos 180° = -1$ $\Rightarrow$ $W = -F \cdot s$ (záporná práce)
\end{itemize}
\end{notebox}

\section{VÝKON}

\subsection{Průměrný výkon}

\begin{definitionbox}
Výkon vyjadřuje, jak rychle se práce koná.
\end{definitionbox}

\begin{equationbox}
\begin{equation*}
    P_p = \frac{W}{t}
\end{equation*}
\end{equationbox}

\begin{notebox}
kde: $P_p$ [W] - průměrný výkon (Watt), $W$ [J] - práce, $t$ [s] - čas
\end{notebox}

\subsection{Okamžitý výkon}

\begin{equationbox}
\begin{equation*}
    P = F \cdot v
\end{equation*}
\end{equationbox}

\begin{notebox}
kde: $P$ [W] - okamžitý výkon, $F$ [N] - síla, $v$ [m$\cdot$s$^{-1}$] - rychlost
\end{notebox}

\newpage

\section{ÚČINNOST}

\begin{definitionbox}
Účinnost je poměr mezi výstupní (užitečnou) a vstupní (dodanou) energií nebo výkonem.
\end{definitionbox}

\begin{equationbox}
\begin{equation*}
    \eta = \frac{P_{\text{výst}}}{P_{\text{vst}}} = \frac{W_{\text{výst}}}{W_{\text{vst}}}
\end{equation*}
\end{equationbox}

\begin{notebox}
kde: $\eta$ - účinnost (řecké písmeno éta) [bezrozměrné číslo 0-1 nebo \%]\\
\textbf{Důležité:} $\eta < 1$ (vždy menší než 100\%)
\end{notebox}

\section{MECHANICKÁ ENERGIE}

\subsection{Kinetická energie (pohybová)}

\begin{definitionbox}
Energie, kterou má těleso díky svému pohybu.
\end{definitionbox}

\begin{equationbox}
\begin{equation*}
    E_k = \frac{1}{2}mv^2
\end{equation*}
\end{equationbox}

\begin{notebox}
kde: $E_k$ [J] - kinetická energie, $m$ [kg] - hmotnost, $v$ [m$\cdot$s$^{-1}$] - rychlost\\
\textbf{Pozor:} Při zdvojnásobení rychlosti se $E_k$ zčtyřnásobí!
\end{notebox}

\subsection{Potenciální energie (polohová)}

\begin{definitionbox}
Energie, kterou má těleso díky své poloze v gravitačním poli.
\end{definitionbox}

\begin{equationbox}
\begin{equation*}
    E_p = mgh
\end{equation*}
\end{equationbox}

\begin{notebox}
kde: $E_p$ [J] - potenciální energie, $m$ [kg] - hmotnost, $g = 9{,}81$ m$\cdot$s$^{-2}$ - tíhové zrychlení, $h$ [m] - výška nad nulovou hladinou
\end{notebox}

\newpage

\section{ZÁKON ZACHOVÁNÍ MECHANICKÉ ENERGIE}

\begin{definitionbox}
V izolované soustavě bez odporových sil se celková mechanická energie nemění.
\end{definitionbox}

\begin{equationbox}
\begin{equation*}
    E_k + E_p = \text{konst.}
\end{equation*}
\end{equationbox}

\begin{equationbox}
\begin{equation*}
    \frac{1}{2}mv^2 + mgh = \text{konst.}
\end{equation*}
\end{equationbox}

\subsection{Rychlost při volném pádu}

\begin{equationbox}
\begin{equation*}
    v = \sqrt{2gh}
\end{equation*}
\end{equationbox}

\begin{notebox}
Odvozeno ze zákona zachování energie: $E_p = E_k$ $\Rightarrow$ $mgh = \frac{1}{2}mv^2$
\end{notebox}

\section{SRÁŽKY TĚLES}

\subsection{Zákon zachování hybnosti}

\begin{definitionbox}
Platí vždy u všech typů srážek.
\end{definitionbox}

\begin{equationbox}
\begin{equation*}
    \vec{p}_{\text{před}} = \vec{p}_{\text{po}}
\end{equation*}
\end{equationbox}

\begin{equationbox}
\begin{equation*}
    m_1\vec{v}_1 + m_2\vec{v}_2 = m_1\vec{v}'_1 + m_2\vec{v}'_2
\end{equation*}
\end{equationbox}

\begin{notebox}
kde: $\vec{p}$ [kg$\cdot$m$\cdot$s$^{-1}$] - hybnost (vektorová veličina!)
\end{notebox}

\subsection{Pružná srážka}

\begin{definitionbox}
Tělesa se od sebe odrazí. Zachovává se hybnost i kinetická energie.
\end{definitionbox}

\begin{equationbox}
\begin{equation*}
    E_{k,\text{před}} = E_{k,\text{po}}
\end{equation*}
\end{equationbox}

\begin{equationbox}
\begin{equation*}
    \frac{1}{2}m_1v_1^2 + \frac{1}{2}m_2v_2^2 = \frac{1}{2}m_1v_1'^2 + \frac{1}{2}m_2v_2'^2
\end{equation*}
\end{equationbox}

\subsection{Nepružná srážka}

\begin{definitionbox}
Tělesa se spojí nebo trvale zdeformují. Zachovává se hybnost, ale nikoliv kinetická energie.
\end{definitionbox}

\begin{equationbox}
\begin{equation*}
    E_{k,\text{před}} > E_{k,\text{po}}
\end{equation*}
\end{equationbox}

\begin{notebox}
Část energie se přemění na: teplo, zvuk, deformaci, vibrace
\end{notebox}

\subsection{Dokonale nepružná srážka}

\begin{definitionbox}
Tělesa se spojí a pohybují společně.
\end{definitionbox}

\begin{equationbox}
\begin{equation*}
    m_1v_1 + m_2v_2 = (m_1 + m_2)v'
\end{equation*}
\end{equationbox}

\begin{notebox}
kde: $v'$ [m$\cdot$s$^{-1}$] - společná rychlost po srážce
\end{notebox}

\newpage

\section{IMPULZ SÍLY}

\begin{definitionbox}
Impulz síly je změna hybnosti tělesa.
\end{definitionbox}

\begin{equationbox}
\begin{equation*}
    I = F \cdot \Delta t = \Delta p = m \cdot \Delta v
\end{equation*}
\end{equationbox}

\begin{notebox}
kde: $I$ [N$\cdot$s] - impulz síly, $\Delta t$ [s] - doba působení síly, $\Delta p$ [kg$\cdot$m$\cdot$s$^{-1}$] - změna hybnosti
\end{notebox}

\section{UŽITEČNÉ VZTAHY}

\subsection{Převody jednotek}

\begin{notebox}
\begin{itemize}
    \item 1 J = 1 N$\cdot$m = 1 kg$\cdot$m$^2\cdot$s$^{-2}$
    \item 1 W = 1 J$\cdot$s$^{-1}$ = 1 N$\cdot$m$\cdot$s$^{-1}$
    \item 1 kWh = 3,6 MJ (1 kilowatthodina = 3,6 megajoulů)
    \item 1 km$\cdot$h$^{-1}$ = 1/3,6 m$\cdot$s$^{-1}$
    \item 1 m$\cdot$s$^{-1}$ = 3,6 km$\cdot$h$^{-1}$
\end{itemize}
\end{notebox}

\subsection{Důležité konstanty}

\begin{notebox}
\begin{itemize}
    \item $g = 9{,}81$ m$\cdot$s$^{-2}$ $\approx$ 10 m$\cdot$s$^{-2}$ (tíhové zrychlení na Zemi)
\end{itemize}
\end{notebox}

\subsection{Praktické příklady účinnosti}

\begin{notebox}
\begin{itemize}
    \item Klasická žárovka: $\eta \approx 5\%$
    \item LED žárovka: $\eta \approx 40\%$
    \item Spalovací motor: $\eta \approx 25-30\%$
    \item Elektromobil: $\eta \approx 85-90\%$
    \item Cyklista: $\eta \approx 20\%$
    \item Zdvihací jeřáb: $\eta \approx 60-70\%$
\end{itemize}
\end{notebox}

\vfill

\begin{center}
\rule{0.8\textwidth}{0.5pt}\\[0.3cm]
{\small Gymnázium Globe | Praktické aplikace fyziky a chemie}
\end{center}

\end{document}
