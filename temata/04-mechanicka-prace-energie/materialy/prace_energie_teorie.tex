\documentclass[11pt,a4paper]{article}
\usepackage[margin=2.5cm]{geometry}
\usepackage[T1]{fontenc}
\usepackage[utf8]{inputenc}
\usepackage[czech]{babel}
\usepackage{amsmath}
\usepackage{booktabs}
\usepackage{longtable}

\title{Mechanická práce a energie - Teoretický přehled}
\author{Fyzika - opakování a prohloubení}
\date{}

\begin{document}
\maketitle

\section{Mechanická práce}

\subsection{Definice}

\textbf{Obecná definice:}
\begin{quote}
Mechanická práce je energie přenesená silou, která působí na těleso po určité dráze.
\end{quote}

\textbf{Rovnice:}
\[W = F \cdot s \cdot \cos\alpha\]

\textbf{Speciální případy:}
\begin{itemize}
\item $\alpha = 0°$ (síla ve směru pohybu): $W = F \cdot s$
\item $\alpha = 90°$ (síla kolmá k pohybu): $W = 0$
\item $\alpha = 180°$ (síla proti pohybu): $W = -F \cdot s$
\end{itemize}

\textbf{Popis veličin:}

\begin{longtable}{lll}
\toprule
Veličina & Popis & Jednotka \\
\midrule
$W$ & Mechanická práce & J (Joule) \\
$F$ & Síla & N \\
$s$ & Dráha & m \\
$\alpha$ & Úhel mezi silou a dráhou & ° nebo rad \\
\bottomrule
\end{longtable}

\textbf{Fyzikální význam:}
\begin{itemize}
\item Práce je skalární veličina (může být kladná, záporná i nulová)
\item Kladná práce: síla koná práci (přidává energii)
\item Záporná práce: síla brzdí (odebírá energii)
\item Nulová práce: síla nekoná práci (kolmá k pohybu)
\end{itemize}

\clearpage

\section{Výkon}

\subsection{Průměrný výkon}

\textbf{Definice:}
\begin{quote}
Průměrný výkon je práce vykonaná za jednotku času.
\end{quote}

\textbf{Rovnice:}
\[P_p = \frac{W}{t}\]

\subsection{Okamžitý výkon}

\textbf{Rovnice:}
\[P = F \cdot v\]

\textbf{Popis veličin:}

\begin{longtable}{lll}
\toprule
Veličina & Popis & Jednotka \\
\midrule
$P$ & Výkon & W (Watt) \\
$W$ & Práce & J \\
$t$ & Čas & s \\
$F$ & Síla & N \\
$v$ & Rychlost & m/s \\
\bottomrule
\end{longtable}

\textbf{Jednotka:}
\begin{itemize}
\item 1 W (Watt) = 1 J/s
\item 1 kW = 1000 W
\item 1 MW = 1\,000\,000 W
\end{itemize}

\subsection{Účinnost}

\textbf{Definice:}
\begin{quote}
Účinnost vyjadřuje poměr užitečné energie (výkonu) k celkové dodané energii (výkonu).
\end{quote}

\textbf{Rovnice:}
\[\eta = \frac{P_{výst}}{P_{vst}} = \frac{W_{výst}}{W_{vst}}\]

\textbf{Vyjádření:}
\begin{itemize}
\item V desetinném tvaru: $0 < \eta < 1$
\item V procentech: $0\% < \eta < 100\%$
\end{itemize}

\textbf{Fyzikální význam:}
\begin{itemize}
\item Vždy $\eta < 1$ (nebo $\eta < 100\%$)
\item Část energie se vždy ztratí (nejčastěji jako teplo třením)
\item Ideální stroj by měl $\eta = 1$ (neexistuje)
\end{itemize}

\clearpage

\section{Mechanická energie}

\subsection{Kinetická energie}

\textbf{Definice:}
\begin{quote}
Kinetická energie je energie pohybujícího se tělesa.
\end{quote}

\textbf{Rovnice:}
\[E_k = \frac{1}{2}mv^2\]

\textbf{Popis veličin:}

\begin{longtable}{lll}
\toprule
Veličina & Popis & Jednotka \\
\midrule
$E_k$ & Kinetická energie & J \\
$m$ & Hmotnost & kg \\
$v$ & Rychlost & m/s \\
\bottomrule
\end{longtable}

\textbf{Fyzikální význam:}
\begin{itemize}
\item Závisí na druhé mocnině rychlosti
\item Zdvojnásobení rychlosti → zčtyřnásobení energie
\item Vždy kladná hodnota (rychlost je na druhou)
\end{itemize}

\subsection{Potenciální energie}

\textbf{Definice:}
\begin{quote}
Potenciální energie je energie polohy tělesa v gravitačním poli.
\end{quote}

\textbf{Rovnice:}
\[E_p = mgh\]

\textbf{Popis veličin:}

\begin{longtable}{lll}
\toprule
Veličina & Popis & Jednotka \\
\midrule
$E_p$ & Potenciální energie & J \\
$m$ & Hmotnost & kg \\
$g$ & Tíhové zrychlení & m/s² \\
$h$ & Výška nad referenční hladinou & m \\
\bottomrule
\end{longtable}

\textbf{Fyzikální význam:}
\begin{itemize}
\item Závisí na výběru nulové hladiny (referenční úrovně)
\item Čím výše, tím větší potenciální energie
\item Při pádu se mění na kinetickou energii
\end{itemize}

\subsection{Celková mechanická energie}

\textbf{Rovnice:}
\[E = E_k + E_p = \frac{1}{2}mv^2 + mgh\]

\clearpage

\section{Zákony zachování}

\subsection{Zákon zachování mechanické energie}

\textbf{Formulace:}
\begin{quote}
V konzervativním systému (bez třecích sil) zůstává celková mechanická energie konstantní.
\end{quote}

\textbf{Rovnice:}
\[E_k + E_p = \text{konst.}\]

\[\frac{1}{2}mv^2 + mgh = \text{konst.}\]

\textbf{Aplikace na volný pád:}

V nejvyšším bodě (h = $h_{max}$, v = 0):
\[E = E_p = mgh_{max}\]

Při dopadu (h = 0, v = $v_{max}$):
\[E = E_k = \frac{1}{2}mv_{max}^2\]

\textbf{Rovnost:}
\[mgh_{max} = \frac{1}{2}mv_{max}^2\]

\textbf{Rychlost při volném pádu:}
\[v = \sqrt{2gh}\]

\subsection{Zákon zachování hybnosti}

\textbf{Formulace:}
\begin{quote}
V izolované soustavě je celková hybnost konstantní.
\end{quote}

\textbf{Rovnice:}
\[\vec{p}_{před} = \vec{p}_{po}\]

Pro dvě tělesa:
\[m_1\vec{v}_1 + m_2\vec{v}_2 = m_1\vec{v}'_1 + m_2\vec{v}'_2\]

\textbf{Aplikace:}
\begin{itemize}
\item Srážky těles
\item Výbuch
\item Raketový pohon
\end{itemize}

\textbf{Důležité:}
\begin{itemize}
\item Platí vždy (i při nepružných srážkách)
\item Zachovává se i v případech, kdy se nezachovává mechanická energie
\end{itemize}

\clearpage

\section{Druhy srážek}

\subsection{Pružná srážka}

\textbf{Charakteristika:}
\begin{itemize}
\item Zachovává se hybnost
\item Zachovává se kinetická energie
\item Dokonale pružné tělesa (idealizace)
\end{itemize}

\textbf{Zákony:}
\[m_1v_1 + m_2v_2 = m_1v'_1 + m_2v'_2\]
\[\frac{1}{2}m_1v_1^2 + \frac{1}{2}m_2v_2^2 = \frac{1}{2}m_1v'^2_1 + \frac{1}{2}m_2v'^2_2\]

\subsection{Nepružná srážka}

\textbf{Charakteristika:}
\begin{itemize}
\item Zachovává se hybnost
\item Nezachovává se kinetická energie (část se mění na teplo, deformaci)
\item Reálné srážky
\end{itemize}

\textbf{Zákon:}
\[m_1v_1 + m_2v_2 = m_1v'_1 + m_2v'_2\]

\textbf{Dokonale nepružná srážka:}
\begin{itemize}
\item Tělesa se po srážce spojí
\item Pohybují se společnou rychlostí
\end{itemize}

\[m_1v_1 + m_2v_2 = (m_1 + m_2)v'\]

\clearpage

\section{Praktické vztahy}

\subsection{Vztah mezi prací a energií}

\textbf{Práce jako změna kinetické energie:}
\[W = \Delta E_k = \frac{1}{2}mv^2 - \frac{1}{2}mv_0^2\]

\textbf{Práce tíhové síly:}
\[W_g = -\Delta E_p = -(mgh_2 - mgh_1) = mg(h_1 - h_2)\]

\subsection{Brzdná dráha}

\textbf{Z kinetické energie:}
\[E_k = W_{tření}\]
\[\frac{1}{2}mv^2 = F_t \cdot s\]
\[s = \frac{mv^2}{2F_t} = \frac{v^2}{2a}\]

Kde $a = F_t/m$ je brzdné zrychlení.

\textbf{Závislost na rychlosti:}
\begin{itemize}
\item Zdvojnásobení rychlosti → zčtyřnásobení brzdné dráhy
\item Proto je rychlost kritická pro bezpečnost
\end{itemize}

\clearpage

\section{Fyzikální konstanty}

\begin{longtable}{llll}
\toprule
Konstanta & Symbol & Hodnota & Jednotka \\
\midrule
Tíhové zrychlení (Země) & $g$ & 9,81 & m/s² \\
Tíhové zrychlení (Měsíc) & $g_M$ & 1,62 & m/s² \\
\bottomrule
\end{longtable}

\clearpage

\section{Souhrn jednotek v SI}

\begin{longtable}{llll}
\toprule
Veličina & Jednotka SI & Odvození & Poznámka \\
\midrule
Práce & J (Joule) & N·m = kg·m²/s² & 1 J = práce síly 1 N po dráze 1 m \\
Energie & J (Joule) & N·m = kg·m²/s² & Stejná jako práce \\
Výkon & W (Watt) & J/s = kg·m²/s³ & 1 W = 1 J/s \\
Účinnost & - & - & Bezrozměrná veličina (0-1 nebo 0-100\%) \\
\bottomrule
\end{longtable}

\textbf{Běžné násobky:}
\begin{itemize}
\item 1 kJ = 1000 J
\item 1 MJ = 1\,000\,000 J
\item 1 kW = 1000 W
\item 1 MW = 1\,000\,000 W
\item 1 kWh = 3\,600\,000 J = 3,6 MJ
\end{itemize}

\clearpage

\section*{Poznámky}

\begin{itemize}
\item \textbf{Konzervativní síla:} Síla, u které práce nezávisí na dráze, ale jen na počáteční a koncové poloze (gravitační síla, elektrická síla)
\item \textbf{Nekonzervativní síla:} Síla, u které práce závisí na dráze (třecí síla)
\item \textbf{Energetická bilance:} Součet všech energií zůstává konstantní (ale může se měnit forma energie)
\item \textbf{Práce a teplo:} Jsou to formy přenosu energie, ne druhy energie
\item \textbf{Perpetuum mobile:} Stroj s účinností 100\% nebo větší - neexistuje (porušoval by termodynamické zákony)
\item \textbf{Vztah práce a výkonu:} Stejná práce může být vykonána s různým výkonem (záleží na čase)
\item \textbf{Elastická potenciální energie:} $E_{pel} = \frac{1}{2}kx^2$ (pružina, kde k je tuhost a x je prodloužení)
\end{itemize}

\end{document}