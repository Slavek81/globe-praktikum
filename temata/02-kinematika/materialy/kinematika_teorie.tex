\documentclass[11pt,a4paper]{article}
\usepackage[margin=2.5cm]{geometry}
\usepackage[T1]{fontenc}
\usepackage[utf8]{inputenc}
\usepackage[czech]{babel}
\usepackage{amsmath}
\usepackage{booktabs}
\usepackage{longtable}

\title{Kinematika - Teoretický přehled}
\author{Fyzika - opakování a prohloubení}
\date{}

\begin{document}
\maketitle

\section{Základní pojmy}

\subsection{Pohyb}

\textbf{Definice:}
\begin{quote}
Pohyb je změna polohy tělesa vůči zvolenému vztažnému systému v čase.
\end{quote}

\textbf{Druhy pohybu:}
\begin{itemize}
\item \textbf{Přímočarý} - trajektorie je přímka
\item \textbf{Křivočarý} - trajektorie je křivka
\item \textbf{Rovnoměrný} - rychlost je konstantní
\item \textbf{Nerovnoměrný} - rychlost se mění
\end{itemize}

\subsection{Dráha a posunutí}

\textbf{Dráha (s):}
\begin{itemize}
\item \textbf{Skalární} veličina
\item Délka trajektorie, kterou těleso urazilo
\item Jednotka: m (metr)
\end{itemize}

\textbf{Posunutí ($\vec{r}$):}
\begin{itemize}
\item \textbf{Vektorová} veličina
\item Přímá vzdálenost mezi počáteční a koncovou polohou
\item Jednotka: m (metr)
\end{itemize}

\subsection{Rychlost}

\textbf{Průměrná rychlost:}
\[v_p = \frac{s}{t}\]

\textbf{Okamžitá rychlost:}
\[v = \lim_{\Delta t \to 0} \frac{\Delta s}{\Delta t} = \frac{ds}{dt}\]

\textbf{Jednotka:} m/s (metr za sekundu)

\subsection{Zrychlení}

\textbf{Definice:}
\begin{quote}
Zrychlení je změna rychlosti v čase.
\end{quote}

\textbf{Rovnice:}
\[a = \frac{\Delta v}{\Delta t} = \frac{v - v_0}{t}\]

\textbf{Jednotka:} m/s² (metr za sekundu na druhou)

\clearpage

\section{Rovnoměrný přímočarý pohyb}

\subsection{Základní rovnice}

\textbf{Dráha:}
\[s = v \cdot t\]

\textbf{Rychlost:}
\[v = \text{konstantní}\]

\textbf{Popis veličin:}

\begin{longtable}{lll}
\toprule
Veličina & Popis & Jednotka \\
\midrule
$s$ & Dráha & m \\
$v$ & Rychlost & m/s \\
$t$ & Čas & s \\
\bottomrule
\end{longtable}

\textbf{Grafické znázornění:}
\begin{itemize}
\item \textbf{Graf s(t):} Přímka procházející počátkem (při $s_0 = 0$)
\item \textbf{Graf v(t):} Vodorovná přímka (konstantní rychlost)
\item \textbf{Graf a(t):} Nulová hodnota (žádné zrychlení)
\end{itemize}

\clearpage

\section{Rovnoměrně zrychlený pohyb}

\subsection{Kinematické rovnice}

\textbf{Rychlost:}
\[v = v_0 + a \cdot t\]

\textbf{Dráha (se začáteční rychlostí):}
\[s = v_0 \cdot t + \frac{1}{2} a \cdot t^2\]

\textbf{Dráha (bez času):}
\[v^2 = v_0^2 + 2 a \cdot s\]

\textbf{Popis veličin:}

\begin{longtable}{lll}
\toprule
Veličina & Popis & Jednotka \\
\midrule
$v$ & Koncová rychlost & m/s \\
$v_0$ & Počáteční rychlost & m/s \\
$a$ & Zrychlení & m/s² \\
$t$ & Čas & s \\
$s$ & Dráha & m \\
\bottomrule
\end{longtable}

\textbf{Grafické znázornění:}
\begin{itemize}
\item \textbf{Graf s(t):} Parabola
\item \textbf{Graf v(t):} Přímka se sklonem $a$
\item \textbf{Graf a(t):} Vodorovná přímka (konstantní zrychlení)
\end{itemize}

\subsection{Speciální případy}

\textbf{Rozjezd z klidu ($v_0 = 0$):}
\[s = \frac{1}{2} a \cdot t^2\]
\[v = a \cdot t\]
\[v^2 = 2 a \cdot s\]

\textbf{Brzdění do zastavení ($v = 0$):}
\[0 = v_0 + a \cdot t \quad \Rightarrow \quad t = -\frac{v_0}{a}\]
\[s = v_0 \cdot t + \frac{1}{2} a \cdot t^2 = \frac{v_0^2}{2|a|}\]

\clearpage

\section{Volný pád}

\subsection{Základní rovnice}

\textbf{Výška:}
\[h = \frac{1}{2} g \cdot t^2\]

\textbf{Rychlost:}
\[v = g \cdot t\]

\textbf{Rychlost bez času:}
\[v^2 = 2 g \cdot h\]

\textbf{Popis veličin:}

\begin{longtable}{lll}
\toprule
Veličina & Popis & Jednotka \\
\midrule
$h$ & Výška & m \\
$v$ & Rychlost & m/s \\
$g$ & Gravitační zrychlení & m/s² \\
$t$ & Čas & s \\
\bottomrule
\end{longtable}

\textbf{Fyzikální význam:}
\begin{itemize}
\item Volný pád je speciální případ rovnoměrně zrychleného pohybu
\item Zrychlení $a = g = 9{,}81$ m/s² (na Zemi)
\item Zanedbáváme odpor vzduchu
\end{itemize}

\clearpage

\section{Vrhy}

\subsection{Svislý vrh vzhůru}

\textbf{Výška:}
\[y = v_0 \cdot t - \frac{1}{2} g \cdot t^2\]

\textbf{Rychlost:}
\[v_y = v_0 - g \cdot t\]

\textbf{Maximální výška:}
\[h_{max} = \frac{v_0^2}{2g}\]

\textbf{Doba výstupu (do maximální výšky):}
\[t_{max} = \frac{v_0}{g}\]

\textbf{Doba letu (celková):}
\[t_{celk} = \frac{2v_0}{g}\]

\subsection{Vodorovný vrh}

\textbf{Vodorovná složka:}
\[x = v_0 \cdot t\]
\[v_x = v_0 = \text{konstantní}\]

\textbf{Svislá složka:}
\[y = -\frac{1}{2} g \cdot t^2\]
\[v_y = -g \cdot t\]

\textbf{Trajektorie:}
\[y = -\frac{g}{2v_0^2} \cdot x^2\] (parabola)

\textbf{Dopad na zem (z výšky $h$):}
\[t = \sqrt{\frac{2h}{g}}\]
\[x = v_0 \cdot \sqrt{\frac{2h}{g}}\]

\subsection{Vrh šikmo vzhůru}

\textbf{Rozklad počáteční rychlosti:}
\[v_x = v_0 \cos \alpha\]
\[v_y = v_0 \sin \alpha\]

\textbf{Pohybové rovnice:}
\[x = v_0 \cos \alpha \cdot t\]
\[y = v_0 \sin \alpha \cdot t - \frac{1}{2} g \cdot t^2\]

\textbf{Trajektorie:}
\[y = x \tan \alpha - \frac{g x^2}{2 v_0^2 \cos^2 \alpha}\]

\textbf{Dosah (R):}
\[R = \frac{v_0^2 \sin 2\alpha}{g}\]

\textbf{Maximální výška:}
\[h_{max} = \frac{v_0^2 \sin^2 \alpha}{2g}\]

\textbf{Doba letu:}
\[t_{celk} = \frac{2v_0 \sin \alpha}{g}\]

\textbf{Popis veličin:}

\begin{longtable}{lll}
\toprule
Veličina & Popis & Jednotka \\
\midrule
$v_0$ & Počáteční rychlost & m/s \\
$\alpha$ & Úhel vrhu & ° nebo rad \\
$v_x, v_y$ & Složky rychlosti & m/s \\
$R$ & Dosah & m \\
$h_{max}$ & Maximální výška & m \\
\bottomrule
\end{longtable}

\clearpage

\section{Pohyb po kružnici}

\subsection{Základní veličiny}

\textbf{Úhlová rychlost:}
\[\omega = \frac{2\pi}{T} = 2\pi f = \frac{\varphi}{t}\]

\textbf{Obvodová rychlost:}
\[v = \omega \cdot r = \frac{2\pi r}{T}\]

\textbf{Perioda:}
\[T = \frac{2\pi}{\omega} = \frac{1}{f}\]

\textbf{Frekvence:}
\[f = \frac{1}{T} = \frac{\omega}{2\pi}\]

\textbf{Popis veličin:}

\begin{longtable}{lll}
\toprule
Veličina & Popis & Jednotka \\
\midrule
$\omega$ & Úhlová rychlost & rad/s \\
$v$ & Obvodová rychlost & m/s \\
$r$ & Poloměr & m \\
$T$ & Perioda & s \\
$f$ & Frekvence & Hz \\
$\varphi$ & Úhel & rad \\
\bottomrule
\end{longtable}

\subsection{Dostředivé zrychlení a síla}

\textbf{Dostředivé zrychlení:}
\[a_d = \frac{v^2}{r} = \omega^2 \cdot r\]

\textbf{Dostředivá síla:}
\[F_d = m \cdot a_d = \frac{m v^2}{r} = m \omega^2 \cdot r\]

\textbf{Fyzikální význam:}
\begin{itemize}
\item Dostředivé zrychlení směřuje k středu kružnice
\item Dostředivá síla je výslednice všech sil, které udržují těleso na kruhové dráze
\item Tato síla neprovádí práci (je kolmá na rychlost)
\end{itemize}

\clearpage

\section{Převody jednotek rychlosti}

\subsection{Základní převody}

\textbf{km/h ↔ m/s:}
\[1 \text{ km/h} = \frac{1000 \text{ m}}{3600 \text{ s}} = \frac{1}{3{,}6} \text{ m/s} \approx 0{,}278 \text{ m/s}\]

\[1 \text{ m/s} = 3{,}6 \text{ km/h}\]

\textbf{Praktický vzorec:}
\[v[\text{m/s}] = \frac{v[\text{km/h}]}{3{,}6}\]
\[v[\text{km/h}] = v[\text{m/s}] \times 3{,}6\]

\subsection{Příklady převodů}

\begin{longtable}{lll}
\toprule
km/h & m/s & Poznámka \\
\midrule
36 & 10 & Běžná rychlost ve městě \\
50 & 13,9 & Rychlost ve městě \\
90 & 25 & Rychlost mimo obec \\
130 & 36,1 & Maximální rychlost na dálnici \\
360 & 100 & Vysoká rychlost \\
\bottomrule
\end{longtable}

\clearpage

\section{Fyzikální konstanty}

\begin{longtable}{llll}
\toprule
Konstanta & Symbol & Hodnota & Jednotka \\
\midrule
Gravitační zrychlení (Země) & $g$ & 9,81 & m/s² \\
Gravitační zrychlení (Měsíc) & $g_M$ & 1,62 & m/s² \\
Rychlost světla ve vakuu & $c$ & 299\,792\,458 & m/s \\
Rychlost zvuku ve vzduchu (20°C) & $v_{zvuk}$ & 343 & m/s \\
\bottomrule
\end{longtable}

\clearpage

\section{Souhrn jednotek v SI}

\begin{longtable}{llll}
\toprule
Veličina & Jednotka SI & Další jednotky & Převody \\
\midrule
Dráha, posunutí & m (metr) & km, cm, mm & 1 km = 1000 m \\
Rychlost & m/s & km/h & 1 m/s = 3,6 km/h \\
Zrychlení & m/s² & - & - \\
Čas & s (sekunda) & min, h & 1 h = 3600 s \\
Úhel & rad (radián) & ° (stupeň) & 1 rad = 57,3° \\
Úhlová rychlost & rad/s & °/s, ot/min & 1 ot/min = 0,105 rad/s \\
Frekvence & Hz (Hertz) & ot/min & 1 Hz = 1/s \\
Perioda & s (sekunda) & min & T = 1/f \\
\bottomrule
\end{longtable}

\clearpage

\section*{Poznámky}

\begin{itemize}
\item \textbf{Trajektorie:} Křivka, po které se těleso pohybuje
\item \textbf{Vztažný systém:} Soustava souřadnic, vůči které měříme pohyb
\item \textbf{Vektor vs. skalár:} Vektor má velikost i směr (posunutí, rychlost, zrychlení), skalár má jen velikost (dráha, rychlost v absolutní hodnotě, čas)
\item \textbf{Pohyb je relativní:} Závisí na volbě vztažného systému (v jednom může být těleso v klidu, v jiném se pohybuje)
\item \textbf{Grafická analýza:} Z grafu s(t) lze určit rychlost (směrnice tečny), z grafu v(t) lze určit zrychlení (směrnice) a dráhu (obsah pod křivkou)
\item \textbf{Optimální úhel vrhu:} Pro maximální dosah je optimální úhel 45° (při vrhu ze stejné výšky na stejnou výšku)
\item \textbf{Kruhový pohyb:} I při konstantní velikosti rychlosti se mění směr → existuje zrychlení (dostředivé)
\end{itemize}

\end{document}